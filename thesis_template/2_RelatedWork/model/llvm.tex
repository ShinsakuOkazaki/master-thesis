LLVM (Low Level Virtual Machine) is an umbrella project which contains components of compilation of programming language.
Existing compilers have tightly coupled functionalities so that it is not possible to embed them into other applications.
However, the abstract framework of LLVM decouples the functionalities into peaces and the peaces of functionalities can be reused.

In structure of a compiler, there are three main components; frontend, optimizer, and backend. 
In frontend, a developer designs the interface of source programming language in way where it can be optimized by optimizer. 
Then, backend takes optimized code and produce the native machine code. LLVM has a component called LLVM Intermediate Representation (IR), 
which places itself across frontend to optimizer. IR is designed to host mid-level analyses and transformations that you find in optimizer section of a compiler.
High-level language has many common structures and functionalities, so most of all high-level program languages can be represented with IR. 
Once source code is represented with IR, optimizer can easily find pattern and optimize it in faster time. 
IR is useful in terms of frontend. This is because developer of language frontend need to know only how the IR works and use the framework to develop a language.