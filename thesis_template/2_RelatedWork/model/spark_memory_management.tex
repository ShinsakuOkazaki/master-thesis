Spark is one of the most used big data computing framework. Spark uses Resilient Distributed Datasets (RDDs) which implement in-memory data structures 
used to cache intermediate data across a set of nodes. This enables multiple rounds of computation on the same data, which is required for machine learning 
and graph analytics iteratively process the data. 

In RDD caching, there are different stages of caching, such as MEMOR\_YONLY and DISK\_ONLY. 
Currently, for very large data sets, we need to pay attention to garbage collection (GC) and OS page swapping overhead, 
because these could degrades execution time significantly. Therefore, DISK\_ONLY RDD caching can be better configuration in this case. 
However, writing and reading intermediate data among desk and memory could have bad effects for execution time, due to need of serialization and deserialization. 