% ABSTRACT

Many of popular open source cluster computing frameworks for large scale data analysis, such as Hadoop and Spark, 
allow programmers to define objects in a host languages, such as Java. The objects are then managed in RAM 
by the language and its runtime, Java Virtual Machine in the case of Java and Scala. Storing objects 
in memory enables machine to process iterative computation. One of the fundamental tasks for recent big data analysis 
is analysis using Machine Learning Algorithms, which require iterative process. As the amount of data increases, memory is required to keep many objects. 
Therefore, memory management plays a critical role in this task. 

Memory management in Java and Scala is performed by garbage collection. The garbage collection brings a significant advantage 
for programmers by removing responsibility for planning memory management by themselves. Instead, JVM monitors the state of memory 
and performs garbage collection at certain points. However, these monitoring and auto-execution of garbage collection cost additional computation 
and might consume computation resources which should be used for data processing. This can significantly decrease performance of the computation. 

In contrast, memory management in system language, such as C++, relies on programmers’ decision for when to allocate and deallocate memory. 
The functions, malloc/free consume most of the memory management. Proper implementation of system language for big data processing can be overperform the implementation in host language. 
Nevertheless, implementing C++ performing proper memory management and guaranteeing security can be unproductive and complicated. 

Considering the issue of memory management, we introduce solution based on unique memory management methods implemented in Rust, ownership and borrowing. 
This unique concepts in Rust secure codes and perform memory management without monitoring memory or calling functions. 
We introduce implementations of machine learning algorithms in both Java and Rust to assess performances of each memory management system for iterative big data processing tasks.

