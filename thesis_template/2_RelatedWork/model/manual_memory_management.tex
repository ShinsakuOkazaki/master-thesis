Programming languages, such as C and C++, perform memory management manuallu and explicitly. The functions, malloc() and free(), take roll for
memory allocation and deallocation in respectively. However, these method to manage memory may cause several problems and require developers 
attention to the problems with siginificant effort for debugging and testing. Here, we explain two of the most common problems regarding to memory management.

Dangling pointer or reference is apointer or reference pointing to object that no longer exists. The situation of dangling pointer happens 
because of deallocation of memory without modification of value of the pointer. 
If the memory region is reallocated for other object and the dangling pointer tries to aceess the original object, the unpredictable behavior may result. 

Memory leak occurs when memory is acllocated and no longer referenced so that the object in the memory location cannot be reached and released.
This is result of dereferencing object with deallocation. Memory leak consumes more memory than necessary by making unreachable location.

Many solutions are established to address these problems. Garbage Collection (GC) is a sofisticated memory management storategy with sacrifice of computation resources 
to track the state of memory. Resource Acquisition is Initialisation (RAII) is a solution without significant overhead and with limitation of situation where is applicable.
Region-Based Memory Management is 


Garbage Collection is automated - Reference Counting, Tracing, the cost of GC is propotional to number of objects (Vector and array is fastor than LinkedList in Java)
(Refer https://www.oracle.com/technetwork/java/javase/tech/memorymanagement-whitepaper-1-150020.pdf)
Resource Acquisition is Initialisation (RAII) (Refer https://www.stephanboyer.com/post/60/region-based-memory-management)
Region-Based Memory Management (Refer Region-Based Memory Management in Cyclone)
Rust, ownership and borrwing 