% This file contains all the necessary setup and commands to create
% the preliminary pages according to the buthesis.sty option.

\title{Big Data Processing for Machine Learning Tasks with Rust}

\author{Shinsaku Okazaki}

% Type of document prepared for this degree:
%   1 = Master of Science thesis,
%   2 = Doctor of Philisophy dissertation.
%   3 = Master of Science thesis and Doctor of Philisophy dissertation.
\degree=2

\prevdegrees{B.S., Seikei University, 2018}

\department{Master of Science}

% Degree year is the year the diploma is expected, and defense year is
% the year the dissertation is written up and defended. Often, these
% will be the same, except for January graduation, when your defense
% will be in the fall of year X, and your graduation will be in
% January of year X+1
\defenseyear{2020}
\degreeyear{2020}

% For each reader, specify appropriate label {First, Second, Third},
% then name, and title. IMPORTANT: The title should be:
%   "Professor of Electrical and Computer Engineering",
% or similar, but it MUST NOT be:
%   Professor, Department of Electrical and Computer Engineering"
% or you will be asked to reprint and get new signatures.
% Warning: If you have more than five readers you are out of luck,
% because it will overflow to a new page. You may try to put part of
% the title in with the name.
\reader{First}{Kia Teymourian, PhD}{Professor of Computer Science}
\reader{Second}{First M. Last}{Associate Professor of \ldots}
\reader{Third}{First M. Last}{Assistant Professor of \ldots}

% The Major Professor is the same as the first reader, but must be
% specified again for the abstract page. Up to 4 Major Professors
% (advisors) can be defined. 
\numadvisors=2
\majorprof{First M. Last, PhD}{{Professor of Electrical and Computer Engineering\\Secondary appointment}}
\majorprofb{First M. Last, PhD}{{Professor of Computer Science}}
%\majorprofc{First M. Last, PhD}{{Professor of Astronomy}}
%\majorprofd{First M. Last, PhD}{{Professor of Biomedical Engineering}}

%%%%%%%%%%%%%%%%%%%%%%%%%%%%%%%%%%%%%%%%%%%%%%%%%%%%%%%%%%%%%%%%  

%                       PRELIMINARY PAGES
% According to the BU guide the preliminary pages consist of:
% title, copyright (optional), approval,  acknowledgments (opt.),
% abstract, preface (opt.), Table of contents, List of tables (if
% any), List of illustrations (if any). The \tableofcontents,
% \listoffigures, and \listoftables commands can be used in the
% appropriate places. For other things like preface, do it manually
% with something like \newpage\section*{Preface}.

% This is an additional page to print a boxed-in title, author name and
% degree statement so that they are visible through the opening in BU
% covers used for reports. This makes a nicely bound copy. Uncomment only
% if you are printing a hardcopy for such covers. Leave commented out
% when producing PDF for library submission.
%\buecethesistitleboxpage

% Make the titlepage based on the above information.  If you need
% something special and can't use the standard form, you can specify
% the exact text of the titlepage yourself.  Put it in a titlepage
% environment and leave blank lines where you want vertical space.
% The spaces will be adjusted to fill the entire page.
\maketitle
\cleardoublepage

% The copyright page is blank except for the notice at the bottom. You
% must provide your name in capitals.
\copyrightpage
\cleardoublepage

% Now include the approval page based on the readers information
\approvalpage
\cleardoublepage

% Here goes your favorite quote. This page is optional.
\newpage
%\thispagestyle{empty}
\phantom{.}
\vspace{4in}

\begin{singlespace}
\begin{quote}
  \textit{Facilis descensus Averni;}\\
  \textit{Noctes atque dies patet atri janua Ditis;}\\*
  \textit{Sed revocare gradum, superasque evadere ad auras,}\\
  \textit{Hoc opus, hic labor est.}\hfill{Virgil (from Don's thesis!)}
\end{quote}
\end{singlespace}

% \vspace{0.7in}
%
% \noindent
% [The descent to Avernus is easy; the gate of Pluto stands open night
% and day; but to retrace one's steps and return to the upper air, that
% is the toil, that the difficulty.]

\cleardoublepage

% The acknowledgment page should go here. Use something like
% \newpage\section*{Acknowledgments} followed by your text.
\newpage
\section*{\centerline{Acknowledgments}}
Here go all your acknowledgments. You know, your advisor, funding agency, lab
mates, etc., and of course your family.

As for me, I would like to thank Jonathan Polimeni for cleaning up old LaTeX
style files and templates so that Engineering students would not have to suffer
typesetting dissertations in MS Word. Also, I would like to thank IDS/ISS
group (ECE) and CV/CNS lab graduates for their contributions and tweaks to this
scheme over the years (after many frustrations when preparing their final
document for BU library). In particular, I would like to thank Limor Martin who
has helped with the transition to PDF-only dissertation format (no more printing
hardcopies -- hooray !!!)

The stylistic and aesthetic conventions implemented in this LaTeX
thesis/dissertation format would not have been possible without the help from
Brendan McDermot of Mugar library and Martha Wellman of CAS.

Finally, credit is due to Stephen Gildea for the MIT style file off which this
current version is based, and Paolo Gaudiano for porting the MIT style to one
compatible with BU requirements.

\vskip 1in

\noindent
Janusz Konrad\\
Professor\\
ECE Department
\cleardoublepage

% The abstractpage environment sets up everything on the page except
% the text itself.  The title and other header material are put at the
% top of the page, and the supervisors are listed at the bottom.  A
% new page is begun both before and after.  Of course, an abstract may
% be more than one page itself.  If you need more control over the
% format of the page, you can use the abstract environment, which puts
% the word "Abstract" at the beginning and single spaces its text.

\begin{abstractpage}
% ABSTRACT


% Motivation 
% - Faster computation and  saving computer resource is important 
% - introduce Big Data tool can be improve
% Problem description
% - introduce Spark, MapReduce is implemented in Application Language.
% - some problems
% Concept 
% - use of rust 
% - what is core concept of experiment
% Evaluation
% - how to evaluate that
% - some notable result

Planning optimized memory management is critical for Big Data analysis tools to perform faster runtime and efficient use of computation resources.
Modern Big Data analysis tools use application languages that abstract their memory management so that developers do not have to pay extreme attention to memory management strategies.

Many existing dataflow tools use Java Virtual Machine (JVM). 
Memory strategies in JVM, such as Garbage Collection (GC), may lead to significant overhead in Big Data processing. 
Apache Spark and Apache Flink use complex objects to manipulate and transfer a tremendous amount of data. 
Generating many of these complex objects in memory forces GC to rearrange the objects in memory frequently waisting computation. 

Considered problems in memory management in JVM, developing Big Data processing tools with system languages can be the solution.
By using a system language, a developer has control on the memory management. Therefore, one can implement systems with more optimized memory management strategies.
We select Rust as a good candidate for the development of Big Data processing tools, due to its ability to write memory-safe and fearless concurrent codes with its concept of ownership.

There may be many possible strategies to optimize memory management for Big Data processing in Rust programming: 
selection of different variable types, use of Reference Counting, and multithreading with Atomic Reference Counting.
We conduct several experiments to assess how much these different memory management strategies differ runtime performance.

Our experiments focus complex object manipulation and common Big Data processing with various memory management.
The results show significant different runtime hits among these different memory strategies.
\end{abstractpage}
\cleardoublepage

% Now you can include a preface. Again, use something like
% \newpage\section*{Preface} followed by your text

% Table of contents comes after preface
\tableofcontents
\cleardoublepage

% If you do not have tables, comment out the following lines
\newpage
\listoftables
\cleardoublepage

% If you have figures, uncomment the following line
\newpage
\listoffigures
\cleardoublepage

% List of Abbrevs is NOT optional (Martha Wellman likes all abbrevs listed)
\chapter*{List of Abbreviations}

{\bf The list below must be in alphabetical order as per BU library instructions or it will be returned to you for re-ordering.}

\begin{center}
  \begin{tabular}{lll}
    \hspace*{2em} & \hspace*{1in} & \hspace*{4.5in} \\
    CAD  & \dotfill & Computer-Aided Design \\
    CO   & \dotfill & Cytochrome Oxidase \\
    DOG  & \dotfill & Difference Of Gaussian (distributions) \\
    FWHM & \dotfill & Full-Width at Half Maximum \\
    LGN  & \dotfill & Lateral Geniculate Nucleus \\
    ODC  & \dotfill & Ocular Dominance Column \\
    PDF  & \dotfill & Probability Distribution Function \\
    $\mathbb{R}^{2}$  & \dotfill & the Real plane \\
  \end{tabular}
\end{center}
\cleardoublepage

% END OF THE PRELIMINARY PAGES

\newpage
\endofprelim
