\documentclass[sigconf]{acmart}

%%
%% \BibTeX command to typeset BibTeX logo in the docs
\AtBeginDocument{%
  \providecommand\BibTeX{{%
    \normalfont B\kern-0.5em{\scshape i\kern-0.25em b}\kern-0.8em\TeX}}}

%% Rights management information.  This information is sent to you
%% when you complete the rights form.  These commands have SAMPLE
%% values in them; it is your responsibility as an author to replace
%% the commands and values with those provided to you when you
%% complete the rights form.
%\setcopyright{acmcopyright}
%\copyrightyear{2018}
%\acmYear{2018}
%\acmDOI{10.1145/1122445.1122456}

%% These commands are for a PROCEEDINGS abstract or paper.
\acmConference[Woodstock '18]{Woodstock '18: ACM Symposium on Neural
  Gaze Detection}{June 03--05, 2018}{Woodstock, NY}
\acmBooktitle{Woodstock '18: ACM Symposium on Neural Gaze Detection,
  June 03--05, 2018, Woodstock, NY}
\acmPrice{15.00}
\acmISBN{978-1-4503-9999-9/18/06}


%%
%% Submission ID.
%% Use this when submitting an article to a sponsored event. You'll
%% receive a unique submission ID from the organizers
%% of the event, and this ID should be used as the parameter to this command.
%%\acmSubmissionID{123-A56-BU3}

%%
%% The majority of ACM publications use numbered citations and
%% references.  The command \citestyle{authoryear} switches to the
%% "author year" style.
%%
%% If you are preparing content for an event
%% sponsored by ACM SIGGRAPH, you must use the "author year" style of
%% citations and references.
%% Uncommenting
%% the next command will enable that style.
%%\citestyle{acmauthoryear}

%%
%% end of the preamble, start of the body of the document source.

\begin{document}

%%
%% The "title" command has an optional parameter,
%% allowing the author to define a "short title" to be used in page headers.
%% \title{Smart Assignment System for Data Science Education}

\title{An Experimental Study of Memory Management in Rust Programming for Big Data Processing}


% \author{Lars Th{\o}rv{\"a}ld}
% \affiliation{%
%  \institution{The Th{\o}rv{\"a}ld Group}
%  \streetaddress{1 Th{\o}rv{\"a}ld Circle}
%  \city{Hekla}
%  \country{Iceland}}
% \email{larst@affiliation.org}

\author{Shinsaku Okazaki}
\affiliation{%
  \institution{Boston University}
}
\email{so4639@bu.edu}


\author{Kia Teymourian}
\affiliation{%
  \institution{Boston University}
}
\email{kiat@bu.edu}





% Other good conference for publising this idea 

% https://iticse.acm.org/2020/




%%
%% The abstract is a short summary of the work to be presented in the
%% article.
\begin{abstract}

Planning optimized memory management is critical for Big Data analysis tools to
perform faster runtime and efficient use of computation resources.
Modern Big Data analysis tools use application languages that abstract their
memory management so that developers do not have to pay extreme attention
to memory management strategies.

Many existing modern cloud-based data processing systems such as Hadoop, Spark or Flink use
Java Virtual Machine (JVM) and taking full advantage of features such as automated memory management in JVM
including Garbage Collection (GC) which may lead to a significant overhead.
Dataflow-based systems like Spark allow programmers to define complex objects in a
host language like Java to manipulate and transfer tremendous amount of data.

Planning optimized memory management is critical for Big Data analysis tools to
perform faster runtime and efficient use of computation resources.
Modern Big Data analysis tools use application languages that abstract their
memory management so that developers do not have to pay extreme attention
to memory management strategies.

Many existing modern cloud-based data processing systems such as Hadoop, Spark or Flink use
Java Virtual Machine (JVM) and taking full advantage of features such as automated memory management in JVM
including Garbage Collection (GC) which may lead to a significant overhead.
Dataflow-based systems like Spark allow programmers to define complex objects in a
host language like Java to manipulate and transfer tremendous amount of data.
\end{abstract}
%%
%% The code below is generated by the tool at http://dl.acm.org/ccs.cfm.
%% Please copy and paste the code instead of the example below.
%%
%\begin{CCSXML}
%<ccs2012>
% <concept>
%  <concept_id>10010520.10010553.10010562</concept_id>
%  <concept_desc>Computer systems organization~Embedded systems</concept_desc>
%  <concept_significance>500</concept_significance>
% </concept>
% <concept>
%  <concept_id>10010520.10010575.10010755</concept_id>
%  <concept_desc>Computer systems organization~Redundancy</concept_desc>
%  <concept_significance>300</concept_significance>
% </concept>
% <concept>
%  <concept_id>10010520.10010553.10010554</concept_id>
%  <concept_desc>Computer systems organization~Robotics</concept_desc>
%  <concept_significance>100</concept_significance>
% </concept>
% <concept>
%  <concept_id>10003033.10003083.10003095</concept_id>
%  <concept_desc>Networks~Network reliability</concept_desc>
%  <concept_significance>100</concept_significance>
% </concept>
%</ccs2012>
%\end{CCSXML}

%\ccsdesc[500]{Computer systems organization~Embedded systems}
%\ccsdesc[300]{Computer systems organization~Redundancy}
%\ccsdesc{Computer systems organization~Robotics}
%\ccsdesc[100]{Networks~Network reliability}

%%
%% Keywords. The author(s) should pick words that accurately describe
%% the work being presented. Separate the keywords with commas.
% \keywords{datasets, neural networks, gaze detection, text tagging}


%%
%% This command processes the author and affiliation and title
%% information and builds the first part of the formatted document.
\maketitle

\section{Introduction}





Our contribution includes: 

\begin{itemize}
  \item 
  \item 
\end{itemize}



\section{Related Work}


\cite{Ahoniemi:2006:AGT:1315803.1315830}




\section{Evaluation}





\section{Conclusion}





\bibliographystyle{ACM-Reference-Format}
\bibliography{mybiblio.bib}


\end{document}
\endinput
