

\subsection{BLAS LAPACK}
\label{sec:linalg_blas_lapack}

Basic Linear Algebra Subprograms (BLAS) \cite{DBLP:journals/toms/LawsonHKK79} are standard building blocks for basic vector and matrix operations. There are 3 levels of operation. The level 1 BLAS performs scalar, 
vector and vector-vector operations, the level 2 BLAS performs matrix-vector operation, and the level 3 BLAS performs matrix-matrix operation. 

LAPACK \cite{laug} is developed on BLAS and has advanced functionalities such as LU decomposition and Singular Value Decomposition (SVD). Dense and banded matrices are handled, but not general 
sparse matrices. The initial motivation of development of LAPACK was to make the widely used EISPACK \cite{DBLP:books/sp/SmithB76} and LINPACK \cite{DBLP:journals/sigarch/Dongarra92} libraries run efficiently on shared-memory vector and parallel processors. 
LINPACK and EISPACK are inefficient because their memory access patterns disregard the multi-layered memory hierarchies of the machines, thereby spending too much time moving data. 
LAPACK addresses this problem by recognizing the algorithms to use block matrix operations, such as matrix multiplication, in the innermost loops. These block operations can be optimized for each architecture to account for the memory hierarchy, 
and so provide a portable way to achieve high efficiency on diverse modern machines. However, LAPACK requires that highly optimized block matrix operations be already implemented on each machine. 

ARPACK \cite{DBLP:books/daglib/0000896} is also a collection of linear algebra subroutines which is designed to compute a few eigenvalues and corresponding eigenvectors from large scale matrix. 
ARPACK is based upon an algorithmic variant of the Arnoldi process called the Implicitly Restarted Arnoldi Method (IRAM). 
When the matrix is symmetric it reduces to a variant of the Lanczos process called the Implicitly Restarted Lanczos Method(IRLM). The Arnoldi process only interacts with the matrix via matrix-vector multiplies. 
Therefore, this method can be applied to distributed matrix operations required in big data analysis.



\subsection{Netlib-Java}
\label{sec:linalg_netlib}
Netlib-java \cite{netlib} is a Java wrapper of BLAS, LAPACK, and ARPACK. Netlib-java choose  implementation of linear algebra depending on installation of the libraries. 
First, if we have installed machine optimized system libraries, such as Intel MKL and OpenBLAS, netlib-java will use these as the implementation to use. 
Next, it try to load netlib references which netlib-java use CBLAS and LAPCKE interface to perform BLAS and LAPACK native call. 
The last option is to use f2j which is intended to translate the BLAS and LAPACK libraries from their Fortran77 reference source code to Java class files, 
instead of calling native libraries by using Java Native Interface (JNI). 

We can use JNI to call native libraries from Java. The JNI is a native programming interface which allows Java code that runs inside a Java Virtual Machine (VM) 
to interoperate with applications and libraries written in other programming languages. 


\subsection{Matrix Computation and Optimization in Apache Spark}
\label{sec:linalg_spark_marix}

Matrix operation is a fundamental part of machine learning. Apache Spark provides implementation for distributed and local matrix operation \cite{DBLP:conf/kdd/ZadehMUYPVSSZ16}. 
To translate single-node algorithms to run on a distributed cluster, Spark addresses separating matrix operations from vector operations and run matrix operations on the cluster, 
while keeping vector operations local to the driver. 

Spark changes its behavior for matrix operations depending on the type of operations and shape of matrices. For example, Singular Value Decomposition (SVD) for a square matrix is performed in distributed cluster, 
but SVD for a tall and skinny matrix is on a driver node. This is because the matrix derived among the computation of SVD for tall and skinny matrix is usually small so that it can fit to single node.

Spark uses ARPACK to solve square SVD. ARPACK is a collection of Fortran77 designed to solve eigenvalue problems. ARPACK is based upon an algorithmic variant of the Arnoldi process called the Implicitly Restarted Arnoldi Method (IRAM). 
When the matrix A is symmetric it reduces to a variant of the Lanczos process called the Implicitly Restarted Lanczos Method (IRLM). 
ARPACK calculate matrix multiplication by performing matrix-vector multiplication. So we can distribute matrix-vector multiplies, and exploit the computational resources available in the entire cluster. 
The other method to distribute matrix operations is Spark TFOCS. Spark TFOCS supports several optimization methods.

To allow full use of hardware-specific linear algebraic operations on single node, Spark uses the BLAS (Basic Linear Algebra Systems) interface with relevant libraries for CPU and GPU acceleration. 
Native libraries can be used in Scala are ones with C BLAS interface or wrapper and called through the Java native interface implemented in Netlib-java library and wrapped by the Scala library called Breeze. 
Following is some of the implementation of BLAS.

\begin{itemize}
    \item f2jblas -  Java implementation of Fortran BLAS
    \item OpenBLAS - open source CPU-optimized C implementation of BLAS
    \item MKL - CPU-optimized C and Fortran implementation of BLAS by Intel
\end{itemize}

These have different implementation and they perform differently for the type of operation and matrices shape. 
In Sark, OpenBlas is the default method of choice. BLAS interface is made specifically for dense linear algebra. 
Then, there are few libraries that efficiently handle sparse matrix operations.

\subsection{Memory Management of each Linear Algebra Library}
\label{sec:linalg_memmanage}

The pure Java linear algebra library, such as La4j, EJML, and Apache Common Math, 
use normal GC performed by JVM to manage memory. This is because the implementation of these libraries are in purely Java.

Netlib-java, Jblas or other simple Java wrapper of BLAS, LAPACK, and ARPACK with Java Native Interface (JNI) use normal GC as well. 
This is because the native code deals with Java array by obtaining a reference to it. After the operation, 
the native method releases the reference to the Java array with or without returning new Java array or Java primitive type object. 

ND4J \cite{nd4j} has two types of  its own memory management methods, GC to pointer of off-heap NDArray, and MemoryWorkspaces. 
ND4J used off-heap memory to store NDArrats, to provide better performance while working with NDArrays from vative code such as BLAS and CUDA libraries. 
Off-heap means that the memory is allocated outside of the Java heap so that it is not managed by the JVM’s GC. NDArray itself is not tracked by JVM, 
but its pointer is. The Java heap stores pointer to NDArray on off-heap. When a pointer is dereferenced, this pointer can be a target of JVM’s GC and when it is collected, 
the corresponding NDArray will be deallocated. When using MemoryWorkspaces, NDArray lives only within specific workspace scope. 
When NDArray leaves the workspace scope, the memory is deallocated unless explicitly calling method to copy the NDArray out of the scope.